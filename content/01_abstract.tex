\pagenumbering{roman}
\setcounter{page}{1}

\selecthungarian

%----------------------------------------------------------------------------
% Abstract in Hungarian
%----------------------------------------------------------------------------
\chapter*{Kivonat}\addcontentsline{toc}{chapter}{Kivonat}

A bioinformatika egy interdiszciplináris terület az informatika és a biológia között. Fő célja biológiai kérdések megválaszolása, melynek első lépése, hogy azokat számítási problémákká alakítja, majd hatékony algoritmikus megoldásokat kínál rájuk. A bioinformatika kutatásának jelentős hatása van mindennapi életünkre, hiszen az ezen a területen elért eredmények segíthetnek megoldani napjaink jelentős globális problémáit: például segíthetik a hatékonyabb gyógyszerek felfedezését, a napjainkban gyakran előforduló különböző genetikai betegségek megértését, olyan ellenálló termények kifejlesztését melyek hozzájárulhatnak a globális éhínség enyhítéséhez vagy új technológiák feltalálását a környezetszennyezés csökkentésére és visszafordítására.

Sajnos a bioinformatika számos gyakorlati fontossággal bíró problémája algoritmuselméleti szempontból nehéz feladatnak bizonyul. Minden kutatási erőfeszítés ellenére ezekre a problémákra nem sikerült kellően gyors, determinisztikus megoldást találni a jelenlegi hardves korlátaink mellett.

Az elmúlt években fokozott figyelem irányult a kvantuminformatika területére mind különböző kormányzati szervek, mind globális nagyvállalatok részéről, akik jelentős támogatással szálltak be az új típusú, kvantumfizikai jelenségeken alapuló hardverek tervezésébe és az ezeken futó, mindeddig kihívást jelentő problémák megoldására alkalmas algoritmusok kifejlesztésébe. Bár a kvantuminformatikai kutatások még gyerekcipőben járnak és a kvantumhardver korlátai sem ismertek, ennek a merőben új számítási modellnek a rendelkezésre állása lehetővé tette teljesen másfajta algoritmustervezési módszerek megjelenését, melyek ígéretes elméleti eredményeket mutattak fel.

Dolgozatomban leírom a kvantumséták matematikai alapjait, különös tekintettel a kvantumsétákra és a kvantumos keresőalgoritmusokra, részletezve a megvalósítás szempontjából fontos pontokat, melyek a szakirodalomban kisebb hangsúllyal szerepelnek. Megvizsgálom a kvantumséták megvalósítási részleteit, és bemutatom azokat egy szimulátor szoftveren keresztül, amely segítségével könnyebben megérthetővé és vizualizálhatóvá válik ezen algoritmusok teljesen újszerű viselkedése. Általános áttekintést adok több bioinformatikai problémáról, különös tekintettel a protein folding (fehérjehajtogatás) problémájára, részletezve a különöző ismert strukturális modelleket, valamint elemzem a probléma klasszikus és kvantumos megoldási lehetőségeit.

\vfill
\selectenglish


%----------------------------------------------------------------------------
% Abstract in English
%----------------------------------------------------------------------------
\chapter*{Abstract}\addcontentsline{toc}{chapter}{Abstract}

Bioinformatics is an interdisciplinary field between computer science and biology. Its main goal is to answer biological questions by transforming them into computational problems and providing efficient algorithmic solutions. Researching bioinformatics significantly impacts our everyday lives, as discoveries in this field could help us solve many of today's major global problems. Utilizing them, we could create novel medical treatments, advance our understanding of genetic diseases, develop resistant crops to tackle a global food crisis or invent new technologies to decrease environmental pollution.

Unfortunately, many practical problems in bioinformatics turn out to be computationally hard ones on current hardware. Despite tremendous research effort, no sufficiently fast, deterministic solutions have been found to these problems.

Quantum informatics is a compelling field for algorithmic research since quantum computers work fundamentally differently from classical ones, and the already established classical problem complexities work differently in the quantum world. One of the most famous examples is Shor's prime factorization algorithm, which could break modern-day encryption quickly on a large-scale quantum computer.

Mainly due to this, governmental entities and global corporations are paying increased attention to quantum computing. They are investing in quantum hardware and software development while funding theoretical research and surrounding communities. The European Commission started Quantum Flagship in 2018, a large-scale 10-year initiative with €1 billion in funding, currently running under the Horizon Europe initiative.

While quantum computing research is still in its early stages and the limits of quantum hardware are unknown, the availability of this different computational model has made new theoretical discoveries possible in both the classical and the quantum worlds.

In my dissertation, I explore the toolkit of quantum algorithms, specifically quantum walks and Grover's search algorithm. I implement visualization software for quantum walks and a generic framework for optimizing memory usage during quantum algorithm simulations. Furthermore, I investigate their possible applications in bioinformatics, such as protein folding. I examine classical and quantum solutions to this problem.

I introduce the mathematical framework of quantum computation, particularly quantum walks and search algorithms. I investigate the implementation details of quantum walks and showcase them via simulator software which can aid in understanding and visualizing the odd behaviour of these algorithms.

I present a general overview of computational bioinformatics problems, particularly protein folding, detailing the different models of protein behaviour and analyzing the classical and quantum possibilities for solutions.

\vfill
\selectthesislanguage

\newcounter{romanPage}
\setcounter{romanPage}{\value{page}}
\stepcounter{romanPage}