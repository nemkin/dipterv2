\pagenumbering{roman}
\setcounter{page}{1}

\selecthungarian

%----------------------------------------------------------------------------
% Abstract in Hungarian
%----------------------------------------------------------------------------
\chapter*{Kivonat}\addcontentsline{toc}{chapter}{Kivonat}

A bioinformatika egy interdiszciplináris terület az informatika és a biológia határán. Fő célja biológiai kérdések megválaszolása, melynek első lépése, hogy azokat számítási problémákká alakítja, majd hatékony algoritmikus megoldásokat kínál rájuk. A bioinformatika kutatásának jelentős hatása van mindennapi életünkre, hiszen az ezen a területen elért eredmények segíthetnek megoldani napjaink jelentős globális problémáit: például lehetővé tehetik a hatékonyabb gyógyszerek felfedezését, a különböző genetikai betegségek megértését, olyan ellenálló termények kifejlesztését melyek hozzájárulhatnak a globális éhínség enyhítéséhez vagy új technológiák feltalálását a környezetszennyezés csökkentésére.

Sajnos a bioinformatika számos gyakorlati fontossággal bíró problémája algoritmuselméleti szempontból nehéz feladatnak bizonyul a klasszikus számítógépeken. Minden kutatási erőfeszítés ellenére ezekre a problémákra a mai napig nem sikerült kellően gyors, determinisztikus megoldást találni.

A kvantuminformatika az algoritmuselméleti kutatás szempontjából egy érdekes terület, hiszen a kvantumszámítógépek működése alapvetően más, mint a klasszikusaké, aminek az a következménye, hogy a klasszikus bonyolultságelmélet a kvantumvilágban másképpen alakul. Az egyik leghíresebb példa erre Shor prímfaktorizációs algoritmusa, amely egy nagy méretű kvantumszámítógépen gyorsan feltörhetné a ma elterjedt titkosításokat.

Ennek köszönhetően az elmúlt években fokozott figyelem irányult a kvantuminformatika területére mind különböző kormányzati szervek, mind globális nagyvállalatok részéről, akik jelentős támogatással szálltak be az új típusú, kvantumfizikai jelenségeken alapuló hardverek tervezésébe és a rajtuk futó szoftverek fejlesztésébe. Az Európai Bizottság például 2018-ban elindította a Quantum Flagship nevű nagyszabású, 10 évre szóló, 1 milliárd eurós finanszírozású kezdeményezését a kvantuminformatikai kutatások finanszírozására, amely a Horizont Európa keretében fut.

Bár a kvantuminformatika még mindig gyerekcipőben jár és a kvantumhardverek határai még nem ismertek, ez az újfajta számítási modell már sok új elméleti felfedezést tett lehetővé mind a klasszikus, mind a kvantumvilágban.

Dolgozatomban ismertetek néhány bioinformatikai problémát, különös tekintettel a fehérjehajtogatás feladatára és elmagyarázom Grover keresési algoritmusával és a kvantum sétákkal való kapcsolatát. Ezek után bemutatom a kvantuminformatika eszköztárát, ezen belül a kvantumsétákat és a Grover-féle keresőalgoritmust. Ismertetem a kvantumséták matematikai alapjait, olyan formában, mely a szakirodalomban kevésbé gyakori, melyből a megvalósítás természetes módon következik, majd az általam írt vizualizációs szoftver segítségével demonstrálom a tulajdonságaikat. Áttekintést adok a kvantumalgoritmusokkal való gyakorlati kísérletezés aktuális problémáiról, konkrétan a kvantum fehérjehajtogatással kapcsolatban, majd megtervezek és megvalósítok egy olyan keretrendszert, amely csökkentheti ezen problémákat.

\vfill

\selectenglish

%----------------------------------------------------------------------------
% Abstract in English
%----------------------------------------------------------------------------
\chapter*{Abstract}\addcontentsline{toc}{chapter}{Abstract}

Bioinformatics is an interdisciplinary field between computer science and biology. Its main goal is to answer biological questions by transforming them into computational problems and providing efficient algorithmic solutions. Researching bioinformatics significantly impacts our everyday lives, as discoveries in this field could help us solve many of today's major global problems. Utilizing them, we could create novel medical treatments, advance our understanding of genetic diseases, develop resistant crops to tackle a global food crisis or invent new technologies to decrease environmental pollution.

Unfortunately, many practical problems in bioinformatics turn out to be computationally hard ones on classical hardware. Despite tremendous research effort to date, no sufficiently fast, deterministic solutions have been found to these problems.

Quantum informatics is a compelling field for algorithmic research since quantum computers work fundamentally differently from classical ones, which means that the already established classical problem complexities are different in the quantum world. One of the most famous examples is Shor's prime factorization algorithm, which could break modern-day encryption quickly on a large-scale quantum computer.

Due to this, governmental entities and global corporations are paying increased attention to quantum computing, and they are investing in quantum hardware and software development. For example, the European Commission started Quantum Flagship in 2018, a large-scale 10-year initiative with €1 billion in funding for quantum research, currently running under the Horizon Europe initiative.

While quantum computing is still in its early stages and the limits of quantum hardware are yet unknown, the availability of a different computational model has already made new theoretical discoveries possible in both the classical and the quantum worlds.

In my dissertation, I present a general overview of some computational problems in bioinformatics, particularly protein folding, and explain its connection to Grover's search algorithm and quantum walks. Then, I introduce the toolkit of quantum computation, specifically quantum walks and Grover's search algorithm. I present the mathematical framework for quantum walks, formulated in a way which is less common in literature, from which implementation follows naturally, and demonstrate their characteristics using the visualization software I have written. I describe the current practical problems with experimenting on quantum algorithms, specifically quantum protein folding, then design and implement a framework which can reduce some of these issues.

\vfill
\selectthesislanguage

\afterpage{\null\newpage}

\newcounter{romanPage}
\setcounter{romanPage}{\value{page}}
\stepcounter{romanPage}