\section{Classical random walks}

A classical random walk describe a stohastic process.

\unsure{The sequence can also be regarded as a special category of Markov chain ->Homogenous?}

Classical random walks on graphs can be defined using Markov-chains. Markov-chains
are well explained in \cite{BreimanProbability} and in \cite{XiaReview}.

\change{Breiman egy könyv, de a Xia cikkben jobban le van írva.}

(Információelmélet előadás / a könyvet be lehetne idézni, bár kevésbé szeretik)

\definition{\textbf{Markov-chain}} (First order, discrete-time, discrete-space) A Markov-chain is a sequence of independent random variables from the same distribution, $X_1, X_2, X_3, \dots$, (with value set $A$), that have the Markov property:

\unsure{Hogy kellene az értékkészletet ($A$) jelölni?}

$P(X_k = x_k | X_{k-1} = x_{k-1}, \dots, X_1 = x_1) = P(X_k = x_k | X_{k-1} = x_{k-1})$

$\forall k\geq{}2$ and $x_{1},\dots, x{k}$ from the value set.

\definition{\textbf{Homogenous Markov-chain}} Time invariant, i.e.:
$P(\Phi_k = i | \Phi_{k-1} = j) = p_{j,i}$ $\forall k\geq{}2$, $\forall i,j \in{} A$, which is called the transition probability from $i$ to $j$ and form the transition probability matrix $P$.

This allows us to represent Homogenous Markov-chains as directed graphs.

\definition{\textbf{Distribution of the Markov-chain}} at the $i$th step, the Markov-chain's distribution is the distribution of $X_i$, which is $P(X_i = j)$.

\definition{\textbf{Stacionary distribution}} of the Markov chain is $p_{j} = \lim\limits_{i \to \infty} P(X_i = j)$.

\definition{\textbf{Graph representation of homogenous Markov-chains}}
Graph $G(V,E)$ represents a homogenous Markov-chain $X_1, X_2, X_3, \dots$, (with value set $A$ and transition probability matrix $P$), if $V=A$ and if $p_{j,i} \neq{} 0$, then $E(i,j) = p_{j,i}$ $\forall{}i,j\in{}A$. If $p_{j,i} = 0$, then there is no edge between vertex $i$ and $j$.

\definition{\textbf{Random walk on a graph}}
A random walk on this graph $G$ visits the nodes represented by the Markov-chain: $X_1, X_2, X_3, \dots$. The first node is the start vertex ($X_1$). In the $k$th step we move from vertex $X_k=i$ to vertex $X_{k+1}=j$, with probability $p_{j,i}$.

\definition{\textbf{Hitting time}} $h_{j,i}$ is the expected number of steps before node $j$ is visited in a random walk starting from node $i$.

Recursively:
\[h_{j,i} = \left\{\begin{array}{lr}
1 + \sum\limits_{k\in{}A}p_{j,k}h_{k,i} & \text{if } i\neq{}j\\
0 & \text{if} i=j
\end{array}
\]

\definition{\textbf{Mixing time}} 