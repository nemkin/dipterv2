\section{Classical random walks}

Before introducing quantum computation and specifically quantum walks, I first overview classical random walks, based on the book Probability and Computing, written by Michael Mitzenmacher and Eli Upfal~\cite{MitzenmacherProbability}.

A \textit{random walk} is a stochastic process modeled by a particular type of Markov chain. While a variety of Markov chains exist, in this work, I use the following definition exclusively.

\begin{definition}[Markov chain]

A discrete time stochastic process $X_0, X_1, X_2, \dots$ on a finite state space $A$ is a Markov chain if it has the Markov property:

\begin{align*}
P(X_k = a_k \mid X_{k-1} = a_{k-1}, \dots, X_0 = a_0) = P(X_k = a_k \mid X_{k-1} = a_{k-1}) \addspace \forall a_0,\dots,a_k\in A.
\end{align*}

\end{definition}

Without loss of generality, we can assume, that $A = \{0,1,\dots,n\}$.

If the Markov chain is homogenous (time-invariant), the probability of moving from state $i \in A$ to state $j \in A$ is independent of $k$, and thus can be shortened the following way:

\begin{align*}
P(X_k = j \mid X_{k-1} = i) = p_{j\leftarrow{}i} = p_{j,i} \addspace \forall k \in \mathds{Z}^+.
\end{align*}

Where $p_{j,i}$ is called the \textit{transition probability} between states $i$ and $j$.

The \textit{transition matrix} $\mathbf{P}$ is formed by the transition probabilities.

\begin{align*}
    \mathbf{P}[j,i] = p_{j,i} 
\end{align*}

It follows, that for each column in $\mathbf{P}$, the sum is $1$.

\begin{align*}
    \sum\limits_{j=0}^{n}\mathbf{P}[j,i] = 1 \addspace \forall i \in \{0,\dots,n\}
\end{align*}

Let the \textit{probability distribution} of the process in the $t$-th step be $\pi_t$. Then, $\pi_t$ can be computed from the starting distribution $\pi_0$ using $\mathbf{P}$.

\begin{align*}
\pi_{t} = \mathbf{P}^t\pi_{0}
\end{align*}

The \textit{stationary distribution} ($\pi$) of the Markov chain is a distribution that does not change with a transition, i.e. $\pi = \mathbf{P}\pi$.

Markov chains can be represented using graphs. A directed, weighted graph $G(V,E)$ with weight function $w:E\rightarrow{}[0,1]$ represents a Markov chain, if $V=A$ and $w(i,j) = \mathbf{P}[j,i]$. If $\mathbf{P}[j,i] = 0$, then $\{i,j\} \not\in E$.

A random walk on graph $G$ starts from $X_0 = a_0$ and visits the vertices of the graph according to the states of the Markov-chain: $X_1 = a_1, X_2 = a_2, \dots$ .

Frequently studied characteristics of random walks are \textit{hitting time}~\cite{XiaReview} and \textit{mixing time}~\cite{MitzenmacherProbability}. Informally, hitting time describes how quickly can a vertex be reached from another vertex in the graph, while mixing time expresses how fast the walk reaches the stationary distribution, where the starting vertex can no longer be identified.

\begin{definition}[Hitting time] Let $h_{j,i}$ be the expected number of steps before node $j$ is visited in a random walk starting from node $i$.
Then, $h_{j,i}$ is given by the following recursive formula:
\begin{align*}
    h_{j,i} = \left\{
        \begin{array}{lr}
            1 + \sum\limits_{k\in{}A}p_{j,k}h_{k,i} & \text{if } i\neq{}j\\
            0 & \text{if } i=j
        \end{array}
    \right.
\end{align*}
\end{definition}

\begin{definition}[Mixing time] The smallest time index of the Markov chain, where the total variational distance between the current and the stationary distribution is not greater than a given $\varepsilon$. This measure still depends on the starting distrubiton $\pi_0$, so we take the maximum over all of the possible $\pi_0$ distributions.

\begin{align*}
m(\varepsilon) = \max\limits_{\pi_0}\{\min\{t : \sum\limits_{j=0}^{n}|\pi_t[j] - \pi[j]| \leq \varepsilon\}\}
\end{align*}

\end{definition}