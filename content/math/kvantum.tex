\chapter{Quantum computing}

Algorithms in quantum computing are derived from the postulates of quantum mechanics. These fundamental rules define how a quantum computer operates, and therefore they are essential for any discourse on quantum algorithms.

\section{The postulates of quantum mechanics}

This introduction is based on the following books: Quantum Computing and Communications by Sándor Imre and Ferenc Balázs~\cite{ImreSandor}, Quantum Computing by Mika Hirvensalo~\cite{Hirvensalo} and Quantum Walks and Search Algorithms by Renato Portugal~\cite{Portugal}.

\lineparagraph{Postulate I. State space}

The state of an isolated physical system can be described using a unit length \textit{state vector} in a Hilbert space (i.e. complex linear vector space), or \textit{state space}, equipped with an inner product.

\begin{definition}[Qubit]
A state vector in the 2 dimensional Hilbert space ($H_2$) is a qubit. The base vectors in this space are
\end{definition}

\begin{align*}
\ket{0} = \begin{pmatrix} 1 \\ 0 \end{pmatrix} \text{, and } \ket{1} = \begin{pmatrix} 0 \\ 1 \end{pmatrix}.
\end{align*}

A generic qubit is written in the form

\begin{align*}
a\ket{0} + b\ket{1} = \begin{pmatrix} a \\ b \end{pmatrix}
\end{align*}

where $a, b \in{} \mathds{C}$ and $|a|^2 + |b|^2 = 1$.

\begin{definition}[Superposition]

A quantum system is said to be in \textit{superpotion}, if its state vector is a linear combination of multiple basis states.

For example $a\ket{0} + b\ket{1}$ is in a superposition of the basis states $\ket{0}$ and $\ket{1}$, with probability amplitudes $a$ and $b$.

\end{definition}

\lineparagraph{Postulate II. Evolution}

The time evolution of an isolated physical system is described using unitary transformation, which depends only on the starting and finishing time of the evolution.

A \textit{quantum algorithm} is a sequence of unitary operators applied to an initial state.

\begin{definition}[Unitary matrix]

$\mathbf{U}$ is unitary if $\mathbf{U}^{\dagger} = \mathbf{U}^{-1}$.

\change{Cite this!}

The following definitions are equivalent:

\begin{enumerate}
    \item $\mathbf{U}$'s rows form an orthonormal basis of $\mathds{C}^n$.
    \item $\mathbf{U}$'s columns form an orthonormal basis of $\mathds{C}^n$.
    \item $\mathbf{U}$ is an isometry: it is injective and preserves length.
    \item $\mathbf{U}$ preserves the inner product.
\end{enumerate}

\end{definition}

\lineparagraph{Postulate III. Measurement}

A quantum measurement is defined by a set of measurement operators $\{\mathbf{M}_m\}$, where $m$ stands for the possible results of the measurement. The probability of measuring $m$ if the system is in state $\ket{v}$ is

\begin{align*}
    P(m | \ket{v}) = \bra{v} \mathbf{M}_m^{\dagger} \mathbf{M}_m \ket{v}.
\end{align*}

The state of the system after measuring $m$ is then

\begin{align*}
    \ket{v'} = \frac{\mathbf{M}_m\ket{v}}{\sqrt{\bra{v} \mathbf{M}_m^{\dagger} \mathbf{M}_m \ket{v}}}.
\end{align*}

The set of measurement operators have to satisfy the following \textit{completeness relation}:

\begin{align*}
    \sum\limits_{m}  \mathbf{M}_m^{\dagger} \mathbf{M}_m = I,
\end{align*}

due to

\begin{align*}
    1 = \sum\limits_{m} P(m | \ket{v}) = \sum\limits_{m} \bra{v} \mathbf{M}_m^{\dagger} \mathbf{M}_m \ket{v}.
\end{align*}

\linesubparagraph{Projective measurement}\label{PostulateIIIProjective}

To distinguish a set of orthonormal states $\{\ket{\varphi_m}\}$, the corresponding measurement operators can be produced as $\mathbf{P}_m = \ket{\varphi_m}\bra{\varphi_m}$, with the following properties.

\begin{property}[$\mathbf{P}_m$ is self adjoint]

\begin{align*}
\mathbf{P}_m^{\dagger} = \mathbf{P}_m
\end{align*}

Since

\begin{align*}
\mathbf{P}_m^{\dagger} =
(\ket{\varphi_m}\bra{\varphi_m})^{\dagger} =
\bra{\varphi_m}^{\dagger}\ket{\varphi_m}^{\dagger} =
\ket{\varphi_m}\bra{\varphi_m} =
\mathbf{P}_m.
\end{align*}

\end{property}

\begin{property}[$\mathbf{P}_m$ is a projection]

\begin{align*}
\mathbf{P}_m\mathbf{P}_m = \mathbf{P}_m
\end{align*}

Since
\begin{align*}
\mathbf{P}_m\mathbf{P}_m =
(\ket{\varphi_m}\bra{\varphi_m})(\ket{\varphi_m}\bra{\varphi_m}) =
\ket{\varphi_m}(\bra{\varphi_m}\ket{\varphi_m})\bra{\varphi_m} =
\ket{\varphi_m}1\bra{\varphi_m} =
\ket{\varphi_m}\bra{\varphi_m} =
\mathbf{P}_m.
\end{align*}

\end{property}

\begin{property}[The $\mathbf{P}_m$ are orthogonal]

\begin{align*}
m \neq n \Rightarrow \mathbf{P}_m\mathbf{P}_n = \mathbf{0}
\end{align*}

Since
\begin{align*}
\mathbf{P}_m\mathbf{P}_n =
(\ket{\varphi_m}\bra{\varphi_m})(\ket{\varphi_n}\bra{\varphi_n}) =
\ket{\varphi_m}(\bra{\varphi_m}\ket{\varphi_n})\bra{\varphi_n} =
\ket{\varphi_m}0\bra{\varphi_n} =
\mathbf{0}.
\end{align*}

\end{property}

From these properties follows, that the probability of measuring $m$ in case of a projective measurement is

\begin{align*}
P(m | \ket{v}) =
\bra{v}\mathbf{P}_m^{\dagger}\mathbf{P}_m \ket{v} =
\bra{v}\mathbf{P}_m\mathbf{P}_m\ket{v} = 
\bra{v}\mathbf{P}_m\ket{v} =
\bra{v}\ket{\varphi_m}\bra{\varphi_m}\ket{v} = |\bra{\varphi_m}\ket{v}|^2.
\end{align*}

For example, the value of a qubit can be any unit length vector in $H_2$, however when we measure it, we will receive one of the base vectors of $H_2$. For $a\ket{0} + b\ket{1}$ we measure $0$ with probability $|a|^2$ and $1$ with probability $|b|^2$.

\lineparagraph{Postulate IV. Composite systems}\label{PostulateIV}

The state space of an isolated composite physical system is the \textit{tensor product} of the state spaces of the individual components. The current state vector of the composite system is the \textit{tensor product} of the current state vectors of the individual systems.

If $V_1,\dots,V_n$ are the state spaces of the individual systems, then $V_1 \otimes \dots \otimes V_n$ is the composite state space and for $\ket{v_1} \in V_1, \dots, \ket{v_n} \in V_n$ state vectors, $\ket{v_1} \otimes \dots \otimes \ket{v_n} = \ket{v_1,\dots,v_n}$ is the state vector of the composite system.

\begin{definition}[Tensor product]

The tensor product $\mathbf{A} \otimes \mathbf{B}$ of matrix $\mathbf{A}_{(r \times s)}$ and matrix $\mathbf{B}_{(t \times u)}$ is of size $(rt \times su)$ and is defined as follows:

\begin{align*}
\text{For }
\mathbf{A} = \begin{pmatrix}
    a_{11} & a_{12} & \dots  & a_{1s} \\
    a_{21} & a_{22} & \dots  & a_{2s} \\
    \vdots & \vdots & \ddots & \vdots \\
    a_{r1} & a_{r2} & \ddots & a_{rs}
\end{pmatrix}
\text{, and }
\mathbf{B} = \begin{pmatrix}
    b_{11} & b_{12} & \dots  & b_{1u} \\
    b_{21} & b_{22} & \dots  & b_{2u} \\
    \vdots & \vdots & \ddots & \vdots \\
    b_{t1} & b_{t2} & \ddots & b_{tu}
\end{pmatrix}
\end{align*}

\begin{align*}
\mathbf{A} \otimes \mathbf{B} = \begin{pmatrix}
    a_{11}\mathbf{B} & a_{12}\mathbf{B} & \dots  & a_{1s}\mathbf{B} \\
    a_{21}\mathbf{B} & a_{22}\mathbf{B} & \dots  & a_{2s}\mathbf{B} \\
    \vdots  & \vdots  & \ddots & \vdots  \\
    a_{r1}\mathbf{B} & a_{r2}\mathbf{B} & \ddots & a_{rs}\mathbf{B}
\end{pmatrix}
\end{align*}

\end{definition}

and has the following properties:

\change{Cite this!}

\begin{property}[Associativity]
\begin{align*}
  (\mathbf{A} \otimes \mathbf{B}) \otimes \mathbf{C} = \mathbf{A} \otimes (\mathbf{B} \otimes \mathbf{C})
\end{align*}
\end{property}

\begin{property}[Distributivity over product]
\label{TensorDistributivity}

If the corresponding matrices are compatible, then

\begin{align*}
  (\mathbf{A} \otimes \mathbf{B})(\mathbf{C} \otimes \mathbf{D}) = (\mathbf{A}\mathbf{C}) \otimes (\mathbf{B}\mathbf{D}).
\end{align*}
\end{property}

\begin{definition}[Quantum register]

The composite system of $n$ qubits is a \textit{quantum register}, having the composite state space

\begin{align*}
H_2^{\otimes{}n} = H_2 \otimes H_2 \otimes ... \otimes H_2
\end{align*}

and for $\ket{q_{n-1}} \in H_2, \dots, \ket{q_{0}} \in H_2$ individual state vectors, the composite state vector is

\begin{align*}
\ket{q_{n-1}} \otimes \dots \otimes \ket{q_{0}} = \ket{q_{n-1},\dots,q_0}.
\end{align*}

\end{definition}

\begin{definition}[Entangled state]
Any state consisting of multiple qubits, that is not decomposable, i.e. that can not be written in the form of a composite system of qubits is \textit{entangled}.
\end{definition}

For example, the state $\frac{1}{\sqrt{2}}(\ket{00} + \ket{11})$ is entangled, since it can not be written in the form

\begin{align*}
(a_0\ket{0} + a_1\ket{1})\otimes(b_0\ket{0}+b_1\ket{1}) = a_0b_0\ket{00}+a_0b_1\ket{01}+a_1b_0\ket{10}+a_1b_1\ket{11}
\end{align*}

since that would require from $a_0b_0 = a_1b_1 = \frac{1}{\sqrt{2}}$, for all coefficients to be non-zero and from $a_0b_1 = a_1b_0 = 0$ for either $a_0$ or $b_1$ and either $a_1$ or $b_0$ to be zero, which is a contradiction.