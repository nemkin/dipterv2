\pagenumbering{roman}
\setcounter{page}{1}

\selecthungarian

%----------------------------------------------------------------------------
% Abstract in Hungarian
%----------------------------------------------------------------------------
\chapter*{Kivonat}\addcontentsline{toc}{chapter}{Kivonat}

A bioinformatika egy interdiszciplináris terület az informatika és a biológia között. Fő célja biológiai kérdések megválaszolása, melynek első lépése, hogy azokat számítási problémákká alakítja, majd hatékony algoritmikus megoldásokat kínál rájuk. A bioinformatika kutatásának jelentős hatása van mindennapi életünkre, hiszen az ezen a területen elért eredmények segíthetnek megoldani napjaink jelentős globális problémáit: például segíthetik a hatékonyabb gyógyszerek felfedezését, a napjainkban gyakran előforduló különböző genetikai betegségek megértését, olyan ellenálló termények kifejlesztését melyek hozzájárulhatnak a globális éhínség enyhítéséhez vagy új technológiák feltalálását a környezetszennyezés csökkentésére és visszafordítására.

Sajnos a bioinformatika számos gyakorlati fontossággal bíró problémája algoritmuselméleti szempontból nehéz feladatnak bizonyul. Minden kutatási erőfeszítés ellenére ezekre a problémákra nem sikerült kellően gyors, determinisztikus megoldást találni a jelenlegi hardves korlátaink mellett.

Az elmúlt években fokozott figyelem irányult a kvantuminformatika területére mind különböző kormányzati szervek, mind globális nagyvállalatok részéről, akik jelentős támogatással szálltak be az új típusú, kvantumfizikai jelenségeken alapuló hardverek tervezésébe és az ezeken futó, mindeddig kihívást jelentő problémák megoldására alkalmas algoritmusok kifejlesztésébe. Bár a kvantuminformatikai kutatások még gyerekcipőben járnak és a kvantumhardver korlátai sem ismertek, ennek a merőben új számítási modellnek a rendelkezésre állása lehetővé tette teljesen másfajta algoritmustervezési módszerek megjelenését, melyek ígéretes elméleti eredményeket mutattak fel.

Dolgozatomban leírom a kvantumséták matematikai alapjait, különös tekintettel a kvantumsétákra és a kvantumos keresőalgoritmusokra, részletezve a megvalósítás szempontjából fontos pontokat, melyek a szakirodalomban kisebb hangsúllyal szerepelnek. Megvizsgálom a kvantumséták megvalósítási részleteit, és bemutatom azokat egy szimulátor szoftveren keresztül, amely segítségével könnyebben megérthetővé és vizualizálhatóvá válik ezen algoritmusok teljesen újszerű viselkedése. Általános áttekintést adok több bioinformatikai problémáról, különös tekintettel a protein folding (fehérjehajtogatás) problémájára, részletezve a különöző ismert strukturális modelleket, valamint elemzem a probléma klasszikus és kvantumos megoldási lehetőségeit.

\vfill
\selectenglish


%----------------------------------------------------------------------------
% Abstract in English
%----------------------------------------------------------------------------
\chapter*{Abstract}\addcontentsline{toc}{chapter}{Abstract}

Bioinformatics is an interdisciplinary field between computer science and biology. Its main goal is to answer biological questions by transforming them into computational problems and providing efficient algorithmic solutions to them. Researching bioinformatics has a significant impact on our everyday lives since discoveries in this area can help us solve many of today's major global problems, for example, aiding the creation of more effective medical treatments, advancing our understanding of genetic diseases, developing resistant crops to tackle a global food crisis or inventing novel technologies to reduce and revert environmental pollution.

Unfortunately, many of the applicable problems in bioinformatics turn out to be computationally hard problems. Despite all the research effort, no sufficiently fast, deterministic solutions were found to these problems with the current limitations of our hardware.

In recent years, the field of quantum informatics has experienced increased attention from governmental entities and global corporations, who invested significantly into designing new types of hardware based on quantum physical phenomena and developing complementary software capable of solving previously challenging problems. While quantum computing research is still in its early stages and the limits of quantum hardware are unknown, the availability of this different computational model has allowed new forms of algorithmic design to emerge, with promising theoretical results.

In my dissertation, I introduce the mathematical framework of quantum computation, particularly quantum walks and search algorithms. I investigate the implementation details of quantum walks and showcase them via a simulator software which can aid understanding and visualizing the unfamiliar behaviour of these algorithms. I present a general overview of computational bioinformatics problems, particularly protein folding, detailing the different models of protein behaviour, and analyse the classical and quantum possibilities for solutions.

\vfill
\selectthesislanguage

\newcounter{romanPage}
\setcounter{romanPage}{\value{page}}
\stepcounter{romanPage}