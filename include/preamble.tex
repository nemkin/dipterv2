% Page layout setup
\pagestyle{plain}
\marginsize{35mm}{25mm}{15mm}{15mm}
\setcounter{tocdepth}{3}
\setcounter{secnumdepth}{3}

% Margón túllógó sorok tiltása.
\sloppy
% A fattyú- és árvasorok elkerülése.
\widowpenalty=10000
\clubpenalty=10000

% Kötőjeles szavak elválasztásának engedélyezése
\def\hyph{-\penalty0\hskip0pt\relax}

% Setup hyperref package
\hypersetup{
    % bookmarks=true,          % show bookmarks bar?
    unicode=true,              % non-Latin characters in Acrobat's bookmarks
    pdftitle={\vikcim},        % title
    pdfauthor={\szerzoMeta},   % author
    pdfsubject={\vikdoktipus}, % subject of the document
    pdfcreator={\szerzoMeta},  % creator of the document
    pdfproducer={},            % producer of the document
    pdfkeywords={},            % list of keywords (separate then by comma)
    pdfnewwindow=true,         % links in new window
    colorlinks=true,           % false: boxed links; true: colored links
    linkcolor=black,           % color of internal links
    citecolor=black,           % color of links to bibliography
    filecolor=black,           % color of file links
    urlcolor=black             % color of external links
}

% Set up listings
\definecolor{lightgray}{rgb}{0.95,0.95,0.95}
\lstset{
	basicstyle=\scriptsize\ttfamily, % print whole listing small
	keywordstyle=\color{black}\bfseries, % bold black keywords
	identifierstyle=, % nothing happens
	% default behavior: comments in italic, to change use
	% commentstyle=\color{green}, % for e.g. green comments
	stringstyle=\scriptsize,
	showstringspaces=false, % no special string spaces
	aboveskip=3pt,
	belowskip=3pt,
	backgroundcolor=\color{lightgray},
	columns=flexible,
	keepspaces=true,
	escapeinside={(*@}{@*)},
	captionpos=b,
	breaklines=true,
	frame=single,
	float=!ht,
	tabsize=2,
	literate=*
		{á}{{\'a}}1	{é}{{\'e}}1	{í}{{\'i}}1	{ó}{{\'o}}1	{ö}{{\"o}}1	{ő}{{\H{o}}}1	{ú}{{\'u}}1	{ü}{{\"u}}1	{ű}{{\H{u}}}1
		{Á}{{\'A}}1	{É}{{\'E}}1	{Í}{{\'I}}1	{Ó}{{\'O}}1	{Ö}{{\"O}}1	{Ő}{{\H{O}}}1	{Ú}{{\'U}}1	{Ü}{{\"U}}1	{Ű}{{\H{U}}}1
}

% Set up theorem-like environments

\theoremseparator{.}
\theorembodyfont{\upshape}
\newtheorem{example}{\pelda}
\newtheorem{property}{\tulajdonsag}
\newtheorem{definition}{\definicio}
\newtheorem{theorem}{\tetel}
\numberwithin{example}{chapter}
\numberwithin{property}{chapter}
\numberwithin{definition}{chapter}
\numberwithin{theorem}{chapter}

\newcommand{\addspace}{\hspace{15pt}}
\newcommand{\lineparagraph}[1]{\paragraph{#1}\mbox{}\\}
\newcommand{\linesubparagraph}[1]{\subparagraph{#1}\mbox{}\\}

% Setup captions
\captionsetup[figure]{
	width=.75\textwidth,
	aboveskip=10pt}

\renewcommand{\captionlabelfont}{\bf}
\renewcommand{\captionfont}{\footnotesize\it}

% Hyphenation exceptions
\hyphenation{Shakes-peare Mar-seilles ár-víz-tű-rő tü-kör-fú-ró-gép}

\author{\vikszerzo}
\title{\viktitle}