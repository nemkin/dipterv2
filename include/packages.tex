% thanks to http://tex.stackexchange.com/a/47579/71109
\usepackage{ifxetex}
\usepackage{ifluatex}
\newif\ifxetexorluatex % a new conditional starts as false
\ifnum 0\ifxetex 1\fi\ifluatex 1\fi>0
  \xetexorluatextrue
\fi

\ifxetexorluatex
  \usepackage{fontspec}
\else
  \usepackage[T1]{fontenc}
  \usepackage[utf8]{inputenc}
  \usepackage[lighttt]{lmodern}
\fi

\usepackage[english,magyar]{babel} % Alapértelmezés szerint utoljára definiált nyelv lesz aktív, de később külön beállítjuk az aktív nyelvet.

%\usepackage{cmap}
\usepackage{amsfonts,amsmath,amssymb} % Mathematical symbols.
%\usepackage[ruled,boxed,resetcount,linesnumbered]{algorithm2e} % For pseudocodes. % beware: this is not compatible with LuaLaTeX, see http://tex.stackexchange.com/questions/34814/lualatex-and-algorithm2e
\usepackage{booktabs} % For publication quality tables for LaTeX
\usepackage{graphicx}

%\usepackage{fancyhdr}
%\usepackage{lastpage}

\usepackage{anysize}
%\usepackage{sectsty}
\usepackage{setspace} % For setting line spacing

\usepackage[unicode]{hyperref} % For hyperlinks in the generated document.
\usepackage[pdftex,dvipsnames,table]{xcolor}  % Coloured text etc.
\usepackage{listings} % For source code snippets.

\usepackage[amsmath,thmmarks]{ntheorem} % Theorem-like environments.

\usepackage[hang]{caption}

\singlespacing

\newcommand{\selecthungarian}{
  \selectlanguage{magyar}
  \setlength{\parindent}{2em}
  \setlength{\parskip}{0em}
  \frenchspacing
}

\newcommand{\selectenglish}{
  \selectlanguage{english}
  \setlength{\parindent}{0em}
  \setlength{\parskip}{0.5em}
  \nonfrenchspacing
  \renewcommand{\figureautorefname}{Figure}
  \renewcommand{\tableautorefname}{Table}
  \renewcommand{\partautorefname}{Part}
  \renewcommand{\chapterautorefname}{Chapter}
  \renewcommand{\sectionautorefname}{Section}
  \renewcommand{\subsectionautorefname}{Section}
  \renewcommand{\subsubsectionautorefname}{Section}
}

\usepackage[numbers]{natbib}
\usepackage{xspace}

% Personal
\usepackage{tikz}
\usepackage{physics} % Braket notation
\usepackage{dsfont} % Mathds
\usepackage{xargs} % Use more than one optional parameter in a new commands
\usepackage[colorinlistoftodos,prependcaption,textsize=tiny]{todonotes} % Todo notes.
\newcommandx{\unsure}[2][1=]{\todo[inline,size=\normalsize,linecolor=red,backgroundcolor=red!25,bordercolor=red,#1]{#2}}
\newcommandx{\change}[2][1=]{\todo[inline,size=\normalsize,linecolor=blue,backgroundcolor=blue!25,bordercolor=blue,#1]{#2}}
\newcommandx{\info}[2][1=]{\todo[inline,size=\normalsize,linecolor=OliveGreen,backgroundcolor=OliveGreen!25,bordercolor=OliveGreen,#1]{#2}}
\usepackage{float} % Place images correctly
